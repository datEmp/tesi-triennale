\documentclass[a4paper]{article}
\usepackage[utf8]{inputenc}
\usepackage[italian]{babel}
\usepackage{import}
\usepackage{graphicx}
\usepackage{float}
\usepackage[margin=1.2in]{geometry}
\usepackage[12pt]{extsizes}
\usepackage{color}
\usepackage{listings}
\usepackage[bottom,flushmargin,hang,multiple]{footmisc}
\usepackage{titlesec}
\usepackage{url}
\usepackage{upquote}
\definecolor{purple}{RGB}{91, 72, 181}
\definecolor{darkviolet}{rgb}{0.58, 0.0, 0.83}
\definecolor{royalblue}{RGB}{242, 73, 214}
\definecolor{orange}{RGB}{227, 72, 41}
\definecolor{lightgreen}{RGB}{164, 194, 14}

\definecolor{mediumgray}{rgb}{0.3, 0.4, 0.4}
\definecolor{forestgreen}{rgb}{0.13, 0.55, 0.13}
\definecolor{crimson}{rgb}{0.86, 0.8, 0.24}

\linespread{1.5}

\setcounter{secnumdepth}{4}
\setcounter{tocdepth}{4}

\titleformat{\paragraph}{\normalfont\normalsize\bfseries}{\theparagraph}{1em}{}
\titlespacing*{\paragraph}{0pt}{3.25ex plus 1ex minus .2ex}{1.5ex plus .2ex}

\lstset{aboveskip=30pt,belowskip=30pt}

\lstdefinestyle{JSES6Base}{
  backgroundcolor=\color{white},
  basicstyle=\ttfamily,
  breakatwhitespace=false,
  breaklines=false,
  captionpos=b,
  columns=fullflexible,
  commentstyle=\color{mediumgray}\upshape,
  emph={},
  emphstyle=\color{crimson},
  extendedchars=true,  % requires inputenc
  fontadjust=true,
  frame=single,
  identifierstyle=\color{black},
  keepspaces=true,
  keywordstyle=\color{purple},
  keywordstyle={[2]\color{darkviolet}},
  keywordstyle={[3]\color{royalblue}},
  keywordstyle={[4]\color{orange}},
  keywordstyle={[5]\color{lightgreen}},
  numbers=left,
  numbersep=5pt,
  numberstyle=\tiny\color{black},
  rulecolor=\color{black},
  showlines=true,
  showspaces=false,
  showstringspaces=false,
  showtabs=false,
  stringstyle=\color{forestgreen},
  tabsize=2,
  title=\lstname,
  upquote=true  % requires textcomp
}

\lstdefinestyle{JavaScript}{
  language=JavaScript,
  style=JSES6Base
}

\lstdefinestyle{ES6}{
  language=ES6,
  style=JSES6Base
}

\lstdefinelanguage{JavaScript}{
  morekeywords=[1]{break, continue, delete, else, for, function, if, in,
    new, return, this, typeof, var, void, while, with, render, getElementById},
  % Literals, primitive types, and reference types.
  morekeywords=[2]{false, null, true, boolean, number, undefined,
    Array, Boolean, Date, Math, Number, String, Object, onClick},
  morekeywords=[3]{h1, div, p, button},
  morekeywords=[4]{React, ., Component, Saluto, Convenevoli, Contatore, setContatore, useState, contatore},
  morekeywords=[5]{},
  sensitive,
  morecomment=[s]{/*}{*/},
  morecomment=[l]//,
  morecomment=[s]{/**}{*/}, % JavaDoc style comments
  morestring=[b]',
  morestring=[b]"
}[keywords, comments, strings]

\lstalias[]{ES6}[ECMAScript2015]{JavaScript}

\lstdefinelanguage[ECMAScript2015]{JavaScript}[]{JavaScript}{
  morekeywords=[1]{await, async, case, catch, class, const, default, do,
    enum, export, extends, finally, from, implements, import, instanceof,
    let, static, super, switch, throw, try},
  morestring=[b]` % Interpolation strings.
}


\title{Progettazione e realizzazione di un exchange decentralizzato per lo scambio di token ERC20 per la piattaforma CommonsHood}

\begin{document}
    \tableofcontents
    \newpage
    \section{Prerequisiti}
    \subsection{Ethereum}
    \subsection{ReactJS}
    React è una libreria JavaScript open-source per lo sviluppo di interfacce utente.
    \subsubsection{Caratteristiche di ReactJS}
    \paragraph{Componenti}
    I Componenti permettono di suddividere la UI in parti indipendenti, riutilizzabili e di pensare ad ognuna di esse in modo isolato.
    Per definire un componente è necessario implementare una funzione JavaScript, ad esempio:

    \begin{lstlisting}[style=ES6, title={Esempio componente}]
        function Saluto(props) {
            return <h1>Ciao, {props.nome}</h1>;
        }\end{lstlisting}

    Questo componente accetta un oggetto parametro contenente dati sotto forma di una singola
    "props", il quale è un oggetto parametro avente dati al suo interno.
    Per renderizzare un componente bisogna utilizzare la funzione \emph{ReactDOM.render()},
    passandole come parametri il componente da visualizzare e il riferimento al componente padre.
    Se si volesse, quindi, renderizzare il componente \emph{Saluto}, passando "\emph{Martina}" come parametro \emph{nome}, il codice potrebbe essere:
    
    \begin{lstlisting}[style=ES6, title={Esempio composizione di componenti}]
      ReactDOM.render(
        <Saluto nome="Martina"/>, 
        document.getElementById('root')
      );\end{lstlisting}
    
    I componenti, inoltre, possono essere composte da altri componenti. In questo caso,
    renderizzando il componente padre, verranno visulizzati anche i componenti figli. Ad esempio, 
    si potrebbe avere un componente \emph{Convenevoli} che contiene multipli componenti \emph{Saluto}:
    \begin{lstlisting}[style=ES6, title={Esempio renderizzazione componente}]
      function Convenevoli() {
        return (
          <div>
            <Saluto nome="Sara" />
            <Saluto nome="Cahal" />
            <Saluto nome="Edite" />
          </div>
        );
      }\end{lstlisting}

      \paragraph{Hook state}
      Un componente di ReactJS di default è stateless. Usando la funzione \emph{useState()} si può
      aggiungere uno stato interno ad un componente, React preserverà questo stato tra le ri-renderizzazioni.
      \emph{useState} ritorna una coppia: il valore dello stato corrente ed una funzione che ci permette di aggiornarlo.
      La funzione ha un unico parametro ed è il suo stato iniziale. Ad esempio, se si volesse realizzare un contatore con un bottone che, alla sua pressione,
      aumenti il valore del contatore, si potrebbe scrivere il seguente codice:
      \begin{lstlisting}[style=ES6, title={Esempio contatore con stato interno}]
        function Contatore() {
          const [contatore, setContatore] = useState(0);
          return (
            <div>
              <p>Hai cliccato {contatore} volte</p>
              <button 
                onClick={() => setContatore(contatore + 1)}>
                Cliccami
              </button>
            </div>
          );
        }
      \end{lstlisting}

      \paragraph{Hook effect}
      Il costrutto \emph{useEffect()} permette l'esecuzione di funzioni ad ogni renderizzazione da parte di React.
      Questa funzione viene utilizzata per effettuare operazioni nei vari stati del ciclo di vita di un componente.
      \newline
      Nel seguente esempio il titolo del documento viene aggiornato all'aumentare del valore del contatore, infatti, 
      ad ogni aggiornamento del DOM da parte di React, viene chiamata la funzione passata a \emph{useEffect()}
      \begin{lstlisting}[style=ES6, title={Esempio uso di useEffect()}]
        function ContatoreConTitolo() {
          const [contatore, setContatore] = useState(0);

          useEffect(() => {
            document.title = `Hai cliccato ${contatore} volte`;
          });

          return (
            <div>
              <p>Hai cliccato {contatore} volte</p>
              <button
                onClick={() => setContatore(contatore + 1)}>
                Cliccami
              </button>
            </div>
          );
        }\end{lstlisting}

        \subsection{ReactJS}
\end{document}